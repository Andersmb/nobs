
	\definesubmol{ZeroOne}{OC -[2] Cr (>:[:20,.85]CO) (<[:-20,.85]CO) (>:[:160,.85]OC) (<[:-160,.85]OC)}
	\definesubmol{OneTwo}{Me >:[:20,.85] = <[:20,.85] Me}
	\definesubmol{TwoTwo}{\emph{t}Bu >:[:20,.85] = <[:20,.85] \emph{t}Bu}
	\definesubmol{ThreeTwo}{Ph >:[:20,.85] = <[:20,.85] Ph}
	\definesubmol{EightOne}{Ni (<[:240] CO) (>:[:200] OC) -[:-30] CO}
\section{Reactions studied}
\subsection{New set}	
	
\begin{figure}[H]
	%\tiny
	\centering
	
	\subfloat[Reaction 0]{
	\setchemfig{scheme debug=false}
	\schemestart
	\chemfig{!{ZeroOne}} \arrow{0}[,0] \+
	\chemfig{=} \arrow
	\chemfig{!{ZeroOne} -[:90,1.25,,,dashed] (=[::90,.5]) (=[::-90,.5])}
	\schemestop
	}
\\
	\subfloat[Reaction 1]{
	\setchemfig{scheme debug=false}
	\schemestart
	\chemfig{!{ZeroOne}} \arrow{0}[,0] \+
	\chemfig{!{OneTwo}} \arrow
	\chemfig{!{ZeroOne} -[:90,1.25,,,dashed] (=[::90,.5] <:[:200,.85] Me) (=[::-90,.5] <[:20,.85] Me) }
	\schemestop
	}
\\
	\subfloat[Reaction 2]{
	\setchemfig{scheme debug=false}
	\schemestart
	\chemfig{!{ZeroOne}} \arrow{0}[,0] \+
	\chemfig{!{TwoTwo}} \arrow
	\chemfig{!{ZeroOne} -[:90,1.25,,,dashed] (=[::90,.5] <:[:200,.85] \emph{t}Bu) (=[::-90,.5] <[:20,.85] \emph{t}Bu) }
	\schemestop
}
\\
\subfloat[Reaction 3]{
	\setchemfig{scheme debug=false}
	\schemestart
	\chemfig{!{ZeroOne}} \arrow{0}[,0] \+
	\chemfig{!{ThreeTwo}} \arrow
	\chemfig{!{ZeroOne} -[:90,1.25,,,dashed] (=[::90,.5] <:[:200,.85] Ph) (=[::-90,.5] <[:20,.85] Ph) }
	\schemestop
}
\\
\subfloat[Reaction 4]{
	\setchemfig{scheme debug=false}
	\schemestart
	\chemfig{!{ZeroOne}} \arrow{0}[,0] \+
	\chemfig{H_2O} \arrow
	\chemfig{!{ZeroOne} -[:90,1.25,,2,dashed] H_2O }
	\schemestop
}
	\caption{Overview of reactions. ``cp$^-$'': deprotonated cyclopentadienyl. ``\emph{m}DMB'': \emph{meta} substituted dimethylbenzene.}
\label{fig: }
\end{figure}

\newpage
\begin{figure}[H]
	%\tiny
	\centering
	\ContinuedFloat
\subfloat[Reaction 5]{
	\setchemfig{scheme debug=false}
	\schemestart
	\chemfig{!{ZeroOne}} \arrow{0}[,0] \+
	\chemfig{MeOH} \arrow
	\chemfig{!{ZeroOne} -[:90,1.25,,2,dashed] MeOH }
	\schemestop
}
\\
\subfloat[Reaction 6]{
	\setchemfig{scheme debug=false}
	\schemestart
	\chemfig{!{ZeroOne}} \arrow{0}[,0] \+
	\chemfig{CH_3C~N} \arrow
	\chemfig{!{ZeroOne} -[:90,1.25,,,dashed] N ~[:90] CCH_3 }
	\schemestop
}
\\
\subfloat[Reaction 7]{
	\setchemfig{scheme debug=false}
	\schemestart
	\chemfig{!{ZeroOne}} \arrow{0}[,0] \+
	\chemfig{THF} \arrow
	\chemfig{!{ZeroOne} -[:90,1.25,,2,dashed] THF  }
	\schemestop
}
\\
\subfloat[Reaction 8]{
	\setchemfig{scheme debug=false}
	\schemestart
	\chemfig{!{EightOne}} \arrow{0}[,0] \+
	\chemfig{=} \arrow
	\chemfig{(=[,.5]) (=[:-180,.5]) -[:-90,,,,dashed] !{EightOne}}
	\schemestop
}
\\
\subfloat[Reaction 9]{
	\setchemfig{scheme debug=false}
	\schemestart
	\chemfig{!{EightOne}} \arrow{0}[,0] \+
	\chemfig{!{OneTwo}} \arrow
	\chemfig{(=[:-180,.5] <:[:200,.85] Me) (=[,.5] <[:20,.85] Me) -[:-90,,,,dashed] !{EightOne}}
	\schemestop
}
	\caption{Continued.}
\label{fig: }
\end{figure}

\newpage
\begin{figure}[H]
	%\tiny
	\centering
	\ContinuedFloat
\subfloat[Reaction 10]{
	\setchemfig{scheme debug=false}
	\schemestart
	\chemfig{!{EightOne}} \arrow{0}[,0] \+
	\chemfig{!{TwoTwo}} \arrow
	\chemfig{(=[:-180,.5] <:[:200,.85] \emph{t}Bu) (=[,.5] <[:20,.85] \emph{t}Bu) -[:-90,,,,dashed] !{EightOne}}
	\schemestop
}
\\
\subfloat[Reaction 11]{
	\setchemfig{scheme debug=false}
	\schemestart
	\chemfig{!{EightOne}} \arrow{0}[,0] \+
	\chemfig{!{ThreeTwo}} \arrow
	\chemfig{(=[:-180,.5] <:[:200,.85] Ph) (=[,.5] <[:20,.85] Ph) -[:-90,,,,dashed] !{EightOne}}
	\schemestop
}
\\
\subfloat[Reaction 12]{
	\setchemfig{scheme debug=false}
	\schemestart
	\chemfig{!{EightOne}} \arrow{0}[,0] \+
	\chemfig{H_2O} \arrow
	\chemfig{H_2O -[:-90,,2,,dashed] !{EightOne}}
	\schemestop
}


\subfloat[Reaction 13]{
	\setchemfig{scheme debug=false}
	\schemestart
	\chemfig{!{EightOne}} \arrow{0}[,0] \+
	\chemfig{MeOH} \arrow
	\chemfig{MeOH -[:-90,,2,,dashed] !{EightOne}}
	\schemestop
}
\\
\subfloat[Reaction 14]{
	\setchemfig{scheme debug=false}
	\schemestart
	\chemfig{!{EightOne}} \arrow{0}[,0] \+
	\chemfig{CH_3C~N} \arrow
	\chemfig{CCH_3~[:-90]N -[:-90,,,,dashed] !{EightOne}}
	\schemestop
}
	\caption{Continued}
\label{fig: }
\end{figure}

\newpage
\begin{figure}[H]
	%\tiny
	\centering
	\ContinuedFloat
\subfloat[Reaction 15]{
	\setchemfig{scheme debug=false}
	\schemestart
	\chemfig{!{EightOne}} \arrow{0}[,0] \+
	\chemfig{THF} \arrow
	\chemfig{THF -[:-90,,2,,dashed] !{EightOne}}
	\schemestop
}
\\
\subfloat[Reaction 16]{
	\setchemfig{scheme debug=false}
	\schemestart
	\chemfig{Cr^{2+}} \arrow{0}[,0] \+
	\chemfig{2cp^{-}} \arrow
	\chemfig{Cr (- cp) (-[4] cp)}
	\schemestop
}
\\
\subfloat[Reaction 17]{
	\setchemfig{scheme debug=false}
	\schemestart
	\chemfig{Cr^{+} - cp^{-}} \arrow{0}[,0] \+
	\chemfig{cp^{-}} \arrow
	\chemfig{Cr (- cp) (-[4] cp)}
	\schemestop
}
\\
\subfloat[Reaction 18]{
	\setchemfig{scheme debug=false}
	\schemestart
	\chemfig{!{ZeroOne}} \arrow{0}[,0] \+
	\chemfig{\emph{m}DMB >:[:20,.85] = <[:20,.85] \emph{m}DMB} \arrow
	\chemfig{!{ZeroOne} -[:90,1.25,,,dashed] (=[::90,.5] <:[:200,.85] \emph{m}DMB) (=[::-90,.5] <[:20,.85] \emph{m}DMB) }
	\schemestop
}
\\
\subfloat[Reaction 19]{
	\setchemfig{scheme debug=false}
	\schemestart
	\chemfig{} \arrow{0}[,0] \+
	\chemfig{MeOH} \arrow
	\chemfig{}
	\schemestop
}
	\caption{Continued}
\label{fig: }
\end{figure}

\newpage
\begin{figure}[H]
	%\tiny
	\centering
	\ContinuedFloat
\subfloat[Reaction 20]{
	\setchemfig{scheme debug=false}
	\schemestart
	\chemfig{!{ZeroOne}} \arrow{0}[,0] \+
	\chemfig{CO} \arrow
	\chemfig{!{ZeroOne} -[:90] CO}
	\schemestop
}
\\
\subfloat[Reaction 21]{
	\setchemfig{scheme debug=false}
	\schemestart
	\chemfig{!{EightOne}} \arrow{0}[,0] \+
	\chemfig{CO} \arrow
	\chemfig{CO -[:-90] !{EightOne}}
	\schemestop
}
\\
\subfloat[Reaction 22]{
	\setchemfig{scheme debug=false}
	\schemestart
	\chemfig{!{EightOne}} \arrow{0}[,0] \+
	\chemfig{\emph{m}DMB >:[:20,.85] = <[:20,.85] \emph{m}DMB} \arrow
	\chemfig{(=[:-180,.5] <:[:200,.85] \emph{m}DMB) (=[,.5] <[:20,.85] \emph{m}DMB) -[:-90,,,,dashed]  !{EightOne}}
	\schemestop
	}
	\caption{Continued}
	\label{fig: }
\end{figure}
	
\subsection{Old set}
The reaction numbers refer to the same indeces used in \textcite{dohm2018}.
Note that reaction 12b is the same as reaction 12 but using trans-stilbene instead of ethene.